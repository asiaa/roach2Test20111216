\documentclass{article}
\oddsidemargin  0.0in
\evensidemargin 0.0in
\textwidth      6.5in
\usepackage{tabularx}
\usepackage{html}
\title{\textbf{CASPER Library} \\Reference Manual}
\newcommand{\Block}[6]{\section {#1 \emph{(#2)}} \label{#3} \textbf{Block Author}: #4 \\ \textbf{Document Author}: #5 \subsection*{Summary}#6}

\newenvironment{PortTable}{\subsection*{Ports}
\tabularx{6.5in}{|l|l|l|X|} \hline  \textbf{Port} & \textbf{Dir.} & \textbf{Data Type} & \textbf{Description} \\ \hline}{\endtabularx}

\newcommand{\Port}[4]{\emph{#1} & \lowercase{#2} & #3 & #4\\  \hline}

\newcommand{\BlockDesc}[1]{\subsection*{Description}#1}

\newenvironment{ParameterTable}{\subsection*{Mask Parameters}
\tabularx{6.5in}{|l|l|X|} \hline  \textbf{Parameter} & \textbf{Variable} & \textbf{Description} \\ \hline}{\endtabularx}

\newcommand{\Parameter}[3]{#1 & \emph{#2} & #3 \\ \hline}

\begin{htmlonly}
\newcommand{\tabularx}[3]{\begin{tabularx}{#1}{#2}{#3}}
\newcommand{\endtabularx}{\end{tabularx}}
\end{htmlonly}

\date{Last Updated \today}
\begin{document}
\maketitle

%\chapter{System Blocks}
%%%%Change Chapter%%%%%%%%
%\chapter{Signal Processing Blocks}

%\input{test.tex}
%\chapter{Communication Blocks}
%\end{document} 
\Block{The Freeze Counter Block}{freeze\_cntr}{freezecntr}{Aaron Parsons}{Aaron Parsons}{A freeze counter is an enabled counter which holds its final value (regardless of enables) until it is reset.}

\begin{ParameterTable}
\Parameter{Counter Length (2$^?$) }{CounterBits}{Specifies the number of bits (and the final count output of $2^{bits-1})$.}
\end{ParameterTable}

\begin{PortTable}
\Port{en}{in}{???}{Step the counter by 1 unless addr=$2^{bits-1}$.}
\Port{rst}{in}{???}{Reset counter to 0.}
\Port{addr}{out}{???}{Current output of the counter.}
\Port{we}{out}{Boolean}{Outputs boolean true just before addr is incremented.}
\Port{done}{out}{Boolean}{Outputs boolean true when a final en is asserted and addr=$2^{bits-1}$.}
\end{PortTable}

\BlockDesc{A freeze counter is an enabled counter which holds its final value (regardless of enables) until it is reset. Thus, a $2^5$ freeze counter will count from 0 to 31 on 31 enables, but will hold 31 thereafter until a reset occurs. This block is useful for writing data in a single pass to memory without looping.} 
\end{document}
