\documentclass{article}
\oddsidemargin  0.0in
\evensidemargin 0.0in
\textwidth      6.5in
\usepackage{tabularx}
\usepackage{html}
\title{\textbf{CASPER Library} \\Reference Manual}
\newcommand{\Block}[6]{\section {#1 \emph{(#2)}} \label{#3} \textbf{Block Author}: #4 \\ \textbf{Document Author}: #5 \subsection*{Summary}#6}

\newenvironment{PortTable}{\subsection*{Ports}
\tabularx{6.5in}{|l|l|l|X|} \hline  \textbf{Port} & \textbf{Dir.} & \textbf{Data Type} & \textbf{Description} \\ \hline}{\endtabularx}

\newcommand{\Port}[4]{\emph{#1} & \lowercase{#2} & #3 & #4\\  \hline}

\newcommand{\BlockDesc}[1]{\subsection*{Description}#1}

\newenvironment{ParameterTable}{\subsection*{Mask Parameters}
\tabularx{6.5in}{|l|l|X|} \hline  \textbf{Parameter} & \textbf{Variable} & \textbf{Description} \\ \hline}{\endtabularx}

\newcommand{\Parameter}[3]{#1 & \emph{#2} & #3 \\ \hline}

\begin{htmlonly}
\newcommand{\tabularx}[3]{\begin{tabularx}{#1}{#2}{#3}}
\newcommand{\endtabularx}{\end{tabularx}}
\end{htmlonly}

\date{Last Updated \today}
\begin{document}
\maketitle

%\chapter{System Blocks}
%%%%Change Chapter%%%%%%%%
%\chapter{Signal Processing Blocks}

%\input{test.tex}
%\chapter{Communication Blocks}
%\end{document} 


\Block{Complex 4-bit Multiplier Implemented in Block RAM}{cmult\_4bit\_br}{cmult4bitbr}{Block Author}{Document Author}{Perform a complex multiplication \emph{(a+bi)(c-di)=(ac-bd)+(ad+bc)i}. Implements the logic in Block RAM.}





\begin{ParameterTable}

\Parameter{Multiplier Latency}{mult\_latency}{The latency through a multiplier.}

\Parameter{Add Latency}{add\_latency}{The latency through an adder.}

\end{ParameterTable}



\begin{PortTable}

\Port{a}{in}{Inherited}{The real component of input 1.}

\Port{b}{in}{Inherited}{The imaginary component of input 1.}

\Port{c}{in}{Inherited}{The real component of input 2.}

\Port{d}{in}{Inherited}{The imaginary component of input 2.}

\Port{real}{out}{Inherited}{ac-bd}

\Port{imag}{out}{Inherited}{ad-bc}

\end{PortTable}



\BlockDesc{Perform a complex multiplication \emph{(a+bi)(c-di)=(ac-bd)+(ad+bc)i}. Implements the logic in Block RAM. 

Each 4 bit real multiplier is implemented as a lookup table with 4b+4b=8b of address.}



 
\end{document}
