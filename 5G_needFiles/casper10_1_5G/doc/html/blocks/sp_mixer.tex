\documentclass{article}
\oddsidemargin  0.0in
\evensidemargin 0.0in
\textwidth      6.5in
\usepackage{tabularx}
\usepackage{html}
\title{\textbf{CASPER Library} \\Reference Manual}
\newcommand{\Block}[6]{\section {#1 \emph{(#2)}} \label{#3} \textbf{Block Author}: #4 \\ \textbf{Document Author}: #5 \subsection*{Summary}#6}

\newenvironment{PortTable}{\subsection*{Ports}
\tabularx{6.5in}{|l|l|l|X|} \hline  \textbf{Port} & \textbf{Dir.} & \textbf{Data Type} & \textbf{Description} \\ \hline}{\endtabularx}

\newcommand{\Port}[4]{\emph{#1} & \lowercase{#2} & #3 & #4\\  \hline}

\newcommand{\BlockDesc}[1]{\subsection*{Description}#1}

\newenvironment{ParameterTable}{\subsection*{Mask Parameters}
\tabularx{6.5in}{|l|l|X|} \hline  \textbf{Parameter} & \textbf{Variable} & \textbf{Description} \\ \hline}{\endtabularx}

\newcommand{\Parameter}[3]{#1 & \emph{#2} & #3 \\ \hline}

\begin{htmlonly}
\newcommand{\tabularx}[3]{\begin{tabularx}{#1}{#2}{#3}}
\newcommand{\endtabularx}{\end{tabularx}}
\end{htmlonly}

\date{Last Updated \today}
\begin{document}
\maketitle

%\chapter{System Blocks}
%%%%Change Chapter%%%%%%%%
%\chapter{Signal Processing Blocks}

%\input{test.tex}
%\chapter{Communication Blocks}
%\end{document} 
%%%%%%%%%%%%%%%%%%%%%%%%%%%%%%%%%
%%%Template for documenting CASPER Library Blocks%%%
%%%%%%%%%%%%%%%%%%%%%%%%%%%%%%%%%

\Block{Mixer}{mixer}{mixer}{Aaron Parsons}{Aaron Parsons, Ben Blackman}{Digitally mixes an input signal (which can be several samples in parallel) with an LO of the indicated frequency (which is some fraction of the native FPGA clock rate).}


\begin{ParameterTable}
\Parameter{Frequency Divisions}{freq\_div}{The (power of 2) denominator of the mixing frequency.}
\Parameter{Mixing Frequency}{freq}{The numerator of the mixing frequency.}
\Parameter{Number of Parallel Streams}{nstreams}{The number of samples that arrive in parallel.}
\Parameter{Bit Width}{n\_bits}{The bitwidth of LO samples.}
\Parameter{BRAM Latency}{bram\_latency}{The latency of sin/cos lookup table.}
\Parameter{MULT Latency}{mult\_latency}{The latency of mixing multipliers.}
\end{ParameterTable}

\begin{PortTable}
\Port{sync}{IN}{boolean}{Takes in an impulse the cycle before the \textit{din}s are valid.}
\Port{dinX}{IN}{Fix\_8\_7}{Input X to be mixed and output on \textit{realX} and \textit{imagX}.}
\Port{sync\_out}{OUT}{boolean}{This signal will be high the cycle before the data coming out is valid.}
\Port{realX}{OUT}{Fix\_(n\_bits)\_(n\_bits-1)}{Real output of mixed \textit{dinX}.}
\Port{imagX}{OUT}{Fix\_(n\_bits)\_(n\_bits-1)}{Imaginary output of mixed \textit{dinX}.}
\end{PortTable}

\BlockDesc{
\paragraph{Usage}
\textit{Mixer} mixes the incoming data and produces both real and imaginary outputs.\\*
M = Frequency Divisions\\*
F = Mixing Frequency\\*
M and F must both be integers, and M must be a power of 2. The ratio F/M should equal the ratio f/r where r is the data rate of the sampled signal. For example, an F/M of 3/16 would downmix an 800Msps signal with an LO of 150MHz.}
 
\end{document}
